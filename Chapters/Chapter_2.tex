\chapter{Le format}

\section{Texte, paragraphes, les titres, et les sous titres}
Pour le texte dans tout le rapport, utilisez « Times New Roman », taille 12. Justifiez votre texte et laissez un peu d’espace au début de chaque paragraphe. L’espace entre les lignes et 1,5, et vous devez ajouter des espaces avant et après les paragraphes pour augmenter la lisibilité. 

Pour les titres et les sous-titres, le style proposé par la classe  «report» doit être gardé.

\section{Code source et Algorithmes}
Les packages \verb|algorithm|, \verb|algorithmicx|, \verb|algpseudocode| et \verb|algorithm2e| peuvent être utilisés pour rédiger des algorithmes avec \LaTeX. Veuillez vous référer au lien suivant pour plus de détails sur l'utilisation de ces packages: \url{https://fr.overleaf.com/learn/latex/Algorithms}

L'algorithme \ref{alg:one} illustre un exemple simple d'un algorithme produit à l'aide du package \verb|algorithm2e|.

%% This is needed if you want to add comments in
%% your algorithm with \Comment
\SetKwComment{Comment}{/* }{ */}

\begin{algorithm}[hbt!]
\caption{An algorithm with caption}\label{alg:one}
\KwData{$n \geq 0$}
\KwResult{$y = x^n$}
$y \gets 1$\;
$X \gets x$\;
$N \gets n$\;
\While{$N \neq 0$}{
  \eIf{$N$ is even}{
    $X \gets X \times X$\;
    $N \gets \frac{N}{2} $ \Comment*[r]{This is a comment}
  }{\If{$N$ is odd}{
      $y \gets y \times X$\;
      $N \gets N - 1$\;
    }
  }
}
\end{algorithm}

Pour les codes sources des programmes et afin de bien les afficher, le package \verb|listings| peut être utilisé. Veuillez consulter le lien suivant pour plus de détails: \url{https://fr.overleaf.com/learn/latex/Code_listing}  

Voici l'exemple suivant qui illustre un code java simple affiché à l'aide de package \verb|listings|. Le style utilisé pour formater ce code tel qu'il apparaît est défini dans le préambule du document.

\begin{lstlisting}[language=Java]
class HelloWorldApp {
    public static void main(String[] args) {
        System.out.println("Hello World!"); // Display the string.
        for (int i = 0; i < 100; ++i) {
            System.out.println(i);
        }
    }
}
\end{lstlisting}

\section{Formules mathématiques}
\LaTeX\ est très pratique pour écrire des mathématiques. En fait, cette fonctionnalité est l'un des aspects les plus importants qui font du \LaTeX\ un choix incontournable pour la rédaction de documents techniques. Le lien suivant montre les commandes les plus élémentaires nécessaires pour commencer à écrire des mathématiques à l'aide du \LaTeX\ : \url{https://fr.overleaf.com/learn/latex/Mathematical_expressions}

Voici un exemple:
\begin{equation}
\left(\begin{array}{cc}
a & b\\
c & d
\end{array}\right)
\times 
\left(\begin{array}{c}
e\\
f
\end{array}\right)
=
\left(\begin{array}{c}
ae+bf\\
ce+df
\end{array}\right)
\end{equation}

\section{Les listes}
Il est souvent nécessaire de présenter de l'information sous forme synthétique ou sous forme de séquence. Les listes sont un excellent outil pour présenter ce genre d'information. Celles-ci peuvent être numérotées ou non numérotées. Différents types de listes peuvent être utilisés dans \LaTeX\ :
\begin{itemize}
\item L'environnement \verb|itemize| pour créer des listes non numérotées,
\item L'environnement \verb|enumerate| pour créer des listes numérotées,
\item L'environnement \verb|description| pour créer des listes de description.
\end{itemize} 

Vous pouvez vous référer au lien suivant pour plus de détails sur la composition et la personnalisation des listes dans \LaTeX : \url{https://fr.overleaf.com/learn/latex/Lists}

\noindent
Voici un exemple de liste numérotée:
\begin{enumerate}
    \item Cette liste est créée à l'aide de l'environnement \verb|enumerate|. 
    \begin{enumerate}
        \item Ce style permet de présenter l'information de façon hiérarchisée et en séquence;
    \end{enumerate}
    \item Ce style propose une numérotation alignée à gauche mais un texte indenté.
\end{enumerate}

\noindent
Voici un exemple de liste non numérotée:
\begin{itemize}
    \item Cette liste utilise l'environnement \verb|itemize|.     
    \begin{itemize}
        \item Par défaut, des puces différentes sont définies pour les quatre premiers niveaux hiérarchiques.
    \end{itemize}
    \item Si vous le désirez, vous pouvez changer les puces proposées.
\end{itemize}


\section{Remarques}
\begin{itemize}
    \item Utilisez les chevrons et l'italique pour les termes d'une langue étrangère: par exemple, schéma de conception (« design pattern »).
    
    \item Vous pouvez utiliser l'italique ou le gras pour mettre en évidence des termes.  Toutefois, il convient de les utiliser de manière uniforme, et avec parcimonie.
\end{itemize}
