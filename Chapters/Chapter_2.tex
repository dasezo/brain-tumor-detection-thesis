\chapter{State of the art on Brain Tumor Detection methods}

\section{Introduction}
This chapter reviews the significant literature connected to this study,
which aims to establish a conceptual information system framework for
medical imaging employing MRI brain tumor pictures. The review focuses on
the use of accessible current electronic scanners, computer-based
methodologies, and their application to enhance the MRI brain tumor tissue
analysis with a comparable accuracy to manual analysis methodology.
\section{Literature Review}
\subsection{Contemporary Methods for Tissue Analysis}
The routine examination of tissues is essential for understanding disease progression and diagnosis. With the increasing availability of medical imaging, it provides an unprecedented opportunity to non-invasively visualize tissue samples and diagnose various medical conditions. In order to make this diagnosis, there are three possible ways to perform this examination in the laboratory, including:

\subsubsection{Traditional Microscopic Analysis}

Histopathological examination remains a cornerstone in disease diagnosis, relying on the meticulous preparation and analysis of tissue specimens. The standard workflow encompasses several critical steps:

\begin{enumerate}
  \item \textbf{Fixation}: Immediately after collection, tissue samples are immersed in a fixative, typically 10\% neutral buffered formalin, to preserve cellular structures and prevent degradation \cite{parry2024histology}.

  \item \textbf{Processing and Embedding}: Fixed tissues undergo dehydration through a series of alcohol baths, clearing with xylene, and infiltration with paraffin wax. This process solidifies the tissue, facilitating thin sectioning \cite{leica2012specimen}.

  \item \textbf{Sectioning}: Using a microtome, thin sections (commonly 3--5\,\textmu m) are sliced from the paraffin-embedded tissue blocks and mounted onto glass slides \cite{parry2024histology}.

  \item \textbf{Staining}: Sections are stained, most commonly with hematoxylin and eosin (H\&E), to differentiate cellular components and enhance contrast for microscopic evaluation \cite{leica2012staining}.
\end{enumerate}

Pathologists examine these slides under varying magnifications (e.g., 10$\times$, 20$\times$, 40$\times$, 100$\times$) to assess tissue architecture and identify pathological changes \cite{wikipedia2025histopathology}.


\subsubsection{Medical Imaging Modalities for Tissue Analysis}

Medical imaging plays a pivotal role in the non-invasive diagnosis and evaluation of various medical conditions, including brain tumors. The primary imaging modalities include ultrasound (US), computed tomography (CT), and magnetic resonance imaging (MRI), each with distinct advantages and limitations.

\textbf{Ultrasound:} Ultrasound imaging utilizes high-frequency sound waves to produce real-time images of internal body structures. It is widely used due to its safety, cost-effectiveness, and portability. However, ultrasound has limitations in imaging structures that are deep within the body or surrounded by bone, such as the brain, making it less suitable for detecting small or deep-seated tumors \cite{raybloc2023ultrasound}.

\textbf{Computed Tomography (CT):} CT scans employ ionizing radiation to generate cross-sectional images of the body. They are particularly effective in visualizing bone structures and detecting acute conditions like hemorrhages. CT scans are faster and more accessible than MRI, making them valuable in emergency settings. However, their ability to differentiate soft tissues is inferior to that of MRI, and the use of ionizing radiation poses potential risks with repeated exposure \cite{healthimages2019ct}.

\textbf{Magnetic Resonance Imaging (MRI):} MRI uses strong magnetic fields and radiofrequency pulses to produce detailed images of soft tissues without ionizing radiation. It offers superior contrast resolution, making it the preferred modality for brain imaging and the detection of tumors, strokes, and other neurological conditions. MRI's ability to distinguish between different soft tissues surpasses that of CT, providing more precise information for diagnosis and treatment planning \cite{healthimages2019mri}.

\subsubsection{Computer-Aided Diagnosis Using Digitized Histopathology Slides}

Traditional manual examination of histopathological slides is time-consuming, subject to inter-observer variability, and may lead to inconsistent diagnostic outcomes. To address these challenges, computer-aided diagnosis (\glsxtrshort{cad}) systems have been developed to assist pathologists by providing quantitative analyses of digitized histopathology images.

CAD systems utilize advanced image processing and machine learning techniques to analyze features such as color, texture, and morphological patterns within tissue samples. For instance, Elazab et al. developed a CAD system that grades brain tumors by extracting color and texture features from histopathology images, achieving significant accuracy in tumor classification \cite{elazab2024computer}.

Furthermore, automated classification methods have been applied to whole-slide digital pathology images. Barker et al. proposed an approach that segments whole-slide images into representative tiles and classifies brain tumors using machine learning algorithms, demonstrating high diagnostic accuracy \cite{barker2016automated}.


\subsection{Image processing techniques}

\subsubsection{Image Pre-processing}
The medical images processed contain a great deal of information as
they are usually noisy due to unwanted pixels. it is always necessary to preprocess as a first step in most current image analysis techniques to analyze
the image of the brain tumor where pre-processing aims to improve:

\begin{enumerate}
  \item \textbf{Noise removal}: Removing impulse noise from images is one of the most important concerns in digital image processing, where noise must be removed in a way that preserves the important information of the image. A variety of techniques are used to eliminate and reduce noise in images, including a Gaussian filter, which is used to remove details and noise. It provides positive and enhanced results for noisy images.
  \item \textbf{Enhance contrast}: It is defined as the manipulation and redistributing the image pixels in a linear or non-linear fashion to improve the separation of obscured structural variations in pixel intensity into a more visually differentiable structural distribution. However, there is no universal theory for enhance contrast approach. Digitized images acquired from MRI are typically grayscale images. It is hard to deal with the intensity of a grayscale image straightway \cite{nishu2012quantifying}. The processing of contrast enhancement histograms is one of the most widely used approaches. The processing of histogram includes equalization (An approach that extends the intensity range and improves image contrast.) and normalization (An approach for modifying the series of pixel intensity values according to relative frequencies).
\end{enumerate}

\subsubsection{Segmentation}
Segmentation is the most important part in image processing. Fence
off an entire image into several parts which is something more meaningful
and easier for further process. These several parts that are rejoined will
cover the entire image. Segmentation may also depend on various features
that are contained in the image. It may be either color or texture. Before
denoising an image, it is segmented to recover the original image. The
main motto of segmentation is to reduce the information for easy analysis.
Segmentation is also useful in Image Analysis and Image Compression \cite{yogamangalam2013segmentation}.

\subsubsection{classification}
Although there are many techniques used to identify and classify each
depending on the characteristics they target for detection, for images of
MRI brain tumor, there are three basic characteristics (shape, size, and
color) required for tumor detection. In most cases, the svm algorithm is
used, but more recently, the Convolutional Neural Network approach is
also widely used in medical image processing.

\subsubsection{Detection}
Image detection is a technique that analyzes the picture and finds
items inside it, medical image detection refers to the process of recognizing
medical-related objects that are included within an image. This assists in
establishing the precise placement of multiple tissues as well as the direction
of those tissues.

\section{Related Work}
\label{sec:related-work}


In this section, we review key approaches to brain tumor segmentation and classification that motivated our work. We focus on some representative studies, highlighting their main techniques and results, and then summarize them in Table~\ref{tab:related-work-summary}.

\subsection{Brain Tumor Segmentation and Grading of Lower‑Grade Glioma}
Naser \emph{et al.}~\cite{naser2020glioma} used a U‑Net–based CNN with transfer learning from VGG16 to segment tumors in 110 T1‑FLAIR cases of low‑grade glioma (LGG). They then classified LGG into grade II vs III, achieving 89\,\% accuracy on slice‑level MRI and 95\,\% at the patient level.

\subsection{Wavelet Statistical Texture + \glsxtrshort{rnn} }
Begum \emph{et al.}~\cite{begum2020wavelet} combined optimal wavelet statistical features with an RNN classifier. Their pipeline includes Gaussian filtering for noise removal, with feature selection, RNN classification, and tumor segmentation via a modified region growing algorithm. They reported 95\,\% accuracy.

\subsection{Hybrid CNN + \glsxtrshort{nade}}
Hashem \emph{et al.}~\cite{hashem2020nade} trained two parallel CNNs whose feature outputs are combined via a Neural Autoregressive Distribution Estimator (NADE). This joint distribution aids in tumor shape identification. Using cross‑entropy loss on 3\,064 T1‑weighted images, they achieved 95\,\% accuracy.

\subsection{Hierarchical Transfer Learning with AlexNet \& GoogleNet}
The framework in~\cite{kulkarni2020framework} applies skull stripping and then uses AlexNet to classify tumors into benign vs malignant, followed by GoogleNet to further distinguish malignant into glioma vs meningioma. With data augmentation (flips, rotations), they report:
\begin{itemize}
  \item Benign vs Malignant: precision 93.75\,\%, recall 100\,\%, F1 96\,\%.
  \item Glioma vs Meningioma: precision 95\,\%, recall 100\,\%, F1 97.43\,\%, accuracy 97.50\,\%.
\end{itemize}

\subsection{VGG Block‑wise Fine‑Tuning}
Lee \emph{et al.}~\cite{swati2019transfer} employ VGG19 with a block‑wise fine‑tuning strategy, dividing the network into six blocks and progressively unfreezing from the last block. Evaluated on the same 3\,064 T1‑weighted set, they reach 94.42\,\% accuracy.

\begin{table}[ht]
  \centering
  \caption{Summary of prior methods in brain tumor segmentation and classification}
  \label{tab:related-work-summary}
  \begin{tabular}{p{0.22\textwidth} p{0.24\textwidth} p{0.16\textwidth} p{0.28\textwidth}}
    \hline
    \textbf{Study}                                       & \textbf{Approach} & \textbf{Dataset} & \textbf{Key Results} \\
    \hline
    \vspace{0.1cm} Naser \emph{et al.}                   &
    \vspace{0.1cm} U-Net + VGG16 transfer learning       &
    \vspace{0.1cm} 110 LGG (T1‑FLAIR)                    &
    \vspace{0.1cm} MRI accuracy: 89\%, patient accuracy: 95\%                                                          \\
    \hline
    \vspace{0.1cm} Begum \emph{et al.}                   &
    \vspace{0.1cm} OGSA wavelet + RNN + MRG segmentation &
    \vspace{0.1cm} BraTS2020                             &
    \vspace{0.1cm} 95\% accuracy                                                                                       \\
    \hline
    \vspace{0.1cm} Hashem \emph{et al.}                  &
    \vspace{0.1cm} Hybrid CNNs + NADE                    &
    \vspace{0.1cm} 3,064 T1-weighted images              &
    \vspace{0.1cm} 95\%                                                                                                \\
    \hline
    \vspace{0.1cm} Framework                             &
    \vspace{0.1cm} AlexNet and GoogleNet hierarchy       &
    \vspace{0.1cm} BraTS2020                             &
    \vspace{0.1cm} Benign vs Mal: F1 96\%;
    \vspace{0.1cm} Glioma vs Men: F1 97.43\%, acc 97.50\%                                                              \\
    \hline
    \vspace{0.1cm} Lee \emph{et al.}                     &
    \vspace{0.1cm} VGG19 block-wise fine-tuning          &
    \vspace{0.1cm} 3,064 T1-weighted images              &
    \vspace{0.1cm} 94.42\%                                                                                             \\
    \hline
  \end{tabular}
\end{table}

\newpage\
\section{Conclusion}
In this chapter, we reviewed key approaches to brain tumor segmentation and classification, highlighting the most utilized machine learning and deep learning models for this purpose. We examined studies that informed our work, focusing on their methodologies and results.
