\chapter*{General Introduction}
\addcontentsline{toc}{chapter}{General Introduction}

Brain tumors represent one of the most challenging conditions in modern medicine, with precise diagnosis and classification being crucial for effective treatment planning and patient outcomes.

Although magnetic resonance imaging (MRI) has revolutionized brain tumor visualization, the interpretation of these scans remains a complex, time-consuming task requiring specialized expertise. Radiologists must analyze hundreds of images per patient, identifying subtle patterns that differentiate tumor types and grades—a process susceptible to inter-observer variability and human fatigue. Recent advances in artificial intelligence, particularly in deep learning and machine learning, offer promising solutions to these challenges. Automated systems can potentially enhance diagnostic accuracy, reduce analysis time, and provide objective, reproducible assessments of tumor characteristics. However, developing such systems requires addressing multiple technical complexities, from precise tumor delineation to accurate grading.

In this project, we trained a hybrid model for brain tumor detection and classification from MRI scans using BraTS dataset. Our approach combines the strengths of deep learning-based segmentation with traditional machine learning classification. Specifically, we utilize a (U-Net) architecture to segment tumor regions from MRI images, followed by feature extraction from these segmented areas, then processed by a Support Vector Machine (\glsxtrshort{svm}) classifier to distinguish between low-grade and high-grade gliomas.

The thesis is structured as follows: We begin by introducing essential medical and technical concepts related to brain anatomy and tumor classification. We then explore the theoretical foundations of artificial intelligence, machine learning, and deep learning. Next, we review the state-of-the-art approaches in brain tumor detection. Finally, we present our methodology, results, and a demonstration of our system's practical application.
