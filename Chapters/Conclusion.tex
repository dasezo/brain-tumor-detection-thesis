\chapter*{General Conclusion}
\addcontentsline{toc}{chapter}{General Conclusion}

This thesis presents a hybrid framework for automated brain tumor detection and classification using MRI. By integrating deep learning and classical machine learning techniques, we demonstrate a robust and effective approach to addressing the challenges of brain tumor analysis.

Our framework, implemented using the BraTS2020 dataset, features two primary modules: a U-Net-based segmentation model and a Support Vector Machine classifier. The segmentation model achieved impressive pixel-level accuracy of 99.3\%, a mean Intersection over Union (IoU) of 74.66\%, and an overall Dice coefficient of 58.98\%, while maintaining high specificity (99.79\%) to avoid false positives. The classifier demonstrated strong performance with an overall accuracy of 93.24\%, showing excellent capability in identifying high-grade gliomas (96\% F1-score), which is particularly valuable given their more aggressive nature and need for urgent intervention. A key contribution of this work is the integration of these modules into an end-to-end pipeline, encapsulated in an intuitive, user-friendly demo application that allows medical professionals to upload patient images and rapidly receive diagnostic assistance.

While results are promising, challenges persist. Improving the Dice coefficient for intricate tumor subregions and enhancing classification, especially for low-grade gliomas (83\% F1-score), are key. Future work should optimize segmentation, explore ensemble learning, expand datasets, implement 3D models, utilize attention mechanisms, validate externally, diversify tumor classification, and integrate explainable AI.

In conclusion, this thesis demonstrates the effectiveness of a hybrid framework combining deep learning and classical machine learning techniques for automated brain tumor analysis. The proposed framework advances the state of the art in tumor detection and classification and provides a practical tool that could assist radiologists in clinical settings.
