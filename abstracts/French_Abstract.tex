\chapter*{\hfill Résumé \hfill}

Les tumeurs cérébrales comptent parmi les troubles neurologiques les plus graves, nécessitant un diagnostic précis et rapide. Ce mémoire présente un système automatisé de détection des tumeurs cérébrales basé sur des techniques d'apprentissage profond et d'apprentissage automatique. Un modèle U-Net est entraîné pour la segmentation sémantique des IRM cérébrales (modalité FLAIR), permettant d’identifier les régions tumorales. Des caractéristiques radiomiques extraites — telles que la surface, l’intensité et la forme — sont utilisées pour classer les tumeurs en gliome de haut grade (HGG) ou de bas grade (LGG) à l’aide d’une machine à vecteurs de support (SVM). Le système traite le déséquilibre des classes en utilisant SMOTE et démontre des résultats prometteurs sur la base de données BraTS2020. Cette approche hybride offre un outil pratique pour l’analyse et la classification des tumeurs.

\noindent\rule{\textwidth}{0.2pt} \hspace{0pt}
\textbf{Mots-clés:} Tumeur cérébrale, Apprentissage profond, Apprentissage automatique, IRM, U-Net, Segmentation sémantique, Radiomique, SVM, SMOTE, BraTS.

\noindent\rule{\textwidth}{0.2pt} \hspace{0pt}
