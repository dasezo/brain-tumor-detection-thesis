\chapter*{\hfill Résumé \hfill}

Les tumeurs cérébrales, en particulier les gliomes, posent un défi clinique important, nécessitant à la fois une localisation précise et un classement précis pour guider le traitement. La segmentation précise des régions tumorales est une première étape essentielle, permettant une analyse et une interprétation significatives des zones touchées. Dans ce projet, nous présentons un cadre hybride qui segmente d’abord les régions tumorales dans l’imagerie par résonance magnétique (IRM) du cerveau en utilisant un modèle U-Net formé sur la base de données de segmentation des tumeurs cérébrales (BraTS), puis classe ces régions comme faibleGliomes de grade ou de haut grade avec un modèle SVM (Support Vector Machine) basé sur les caractéristiques extraites des masques segmentés. Sur l’ensemble de test non exécuté, notre U-Net a atteint une précision de 99,3\,\%, tandis que le classificateur SVM a fourni une précision globale de 93\,\%.


\noindent\rule{\textwidth}{0.2pt}
\textbf{Mots-clés:} Tumeur cérébrale, U-Net, SVM, IRM, BraTS, Segmentation.\\
\noindent\rule{\textwidth}{0.2pt}
