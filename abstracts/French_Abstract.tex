\chapter*{\hfill Résumé \hfill}

Les tumeurs cérébrales, en particulier les gliomes, comptent parmi les types de tumeurs cérébrales les plus répandus et les plus agressifs. Un diagnostic précis et une classification des gliomes sont essentiels pour une planification efficace du traitement et une gestion optimale des patients.

Dans cette thèse, nous nous concentrons sur le développement d'un modèle hybride capable de détecter les tumeurs cérébrales et de les classifier à partir d'images IRM. Le modèle est conçu pour effectuer deux tâches principales : la segmentation des tumeurs et la classification de leur grade. En s'appuyant sur la base de données BraTS2020, qui fournit une collection complète d'IRM avec des régions tumorales annotées et leurs grades correspondants, le modèle intègre une architecture U-Net pour une segmentation précise des régions tumorales dans les IRM et un classificateur SVM (Support Vector Machine) pour déterminer les grades des tumeurs en gliomes de haut grade ou de bas grade. Le modèle U-Net est entraîné pour segmenter avec précision les régions tumorales du tissu cérébral sain environnant, tandis que le classificateur SVM est entraîné pour classifier les tumeurs segmentées en fonction des caractéristiques extraites des images segmentées.

\noindent\rule{\textwidth}{0.2pt}
\textbf{Mots-clés:} Tumeur cérébrale, Gliome, U-Net, SVM, IRM, BraTS, Segmentation des tumeurs, Classification des grades.\\
\noindent\rule{\textwidth}{0.2pt}
