\chapter*{\hfill ملخص \hfill}

تُعدّ أورام الدماغ من أخطر الاضطرابات العصبية، حيث تتطلب تشخيصًا دقيقًا وسريعًا. يعرض هذا البحث نظامًا آليًا لاكتشاف أورام الدماغ بالاعتماد على تقنيات التعلم العميق والتعلم الآلي. تم تدريب نموذج عصبي لتقسيم صور الرنين المغناطيسي للدماغ (باستخدام نمط التوهج السائلي)، بهدف تحديد المناطق المصابة بالورم. يتم استخراج خصائص شعاعية مثل المساحة، الشدة، والشكل، وتُستخدم لتصنيف الأورام إلى أورام دبقية عالية الدرجة أو منخفضة الدرجة باستخدام آلة ناقلات الدعم. يعالج النظام مشكلة عدم توازن العينات من خلال تقنية التوليد الاصطناعي، وقد أظهر نتائج مشجعة على قاعدة بيانات خاصة بالأورام الدماغية. تُعد هذه المقاربة المدمجة أداة عملية لتحليل وتصنيف الأورام.

\noindent\rule{\textwidth}{0.2pt}
\textbf{الكلمات المفتاحية:} أورام الدماغ، التعلم العميق، التعلم الآلي، التصوير بالرنين المغناطيسي، التقسيم الدلالي، الخصائص الشعاعية، آلة ناقلات الدعم، التوليد الاصطناعي، قاعدة بيانات الأورام براتس .\\
\noindent\rule{\textwidth}{0.2pt}