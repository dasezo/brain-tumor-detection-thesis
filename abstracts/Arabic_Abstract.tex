\chapter*{\hfill  ملخص \hfill}

تُعد أورام الدماغ، وخصوصًا الأورام الدبقية، من بين أكثر أنواع أورام الدماغ انتشارًا وعدوانية. يُعتبر التشخيص الدقيق وتصنيف الأورام الدبقية أمرًا بالغ الأهمية للتخطيط العلاجي الفعّال وإدارة حالة المرضى.

في هذه الأطروحة، نركز على تطوير نموذج هجين قادر على اكتشاف أورام الدماغ وتصنيفها من صور التصوير بالرنين المغناطيسي. تم تصميم النموذج لتنفيذ مهمتين رئيسيتين: تقسيم الأورام وتصنيف درجاتها. يعتمد النموذج على مجموعة بيانات \textenglish{BraTS2020}، التي توفر مجموعة شاملة من صور الرنين المغناطيسي مع مناطق الأورام المشروحة والدرجات المقابلة لها. يدمج النموذج بين بنية \textenglish{U-Net} لتقسيم دقيق لمناطق الأورام في صور الرنين المغناطيسي ومصنف آلة ناقلات الدعم \textenglish{SVM} لتحديد درجات الأورام كأورام دبقية عالية الدرجة أو منخفضة الدرجة. يتم تدريب نموذج \textenglish{U-Net} على تقسيم مناطق الورم بدقة عن الأنسجة السليمة المحيطة في الدماغ، بينما يتم تدريب مصنف \textenglish{SVM} على تصنيف الأورام المقسمة إلى درجاتها بناءً على الخصائص المستخرجة من الصور المقسمة.

\noindent\rule{\textwidth}{0.2pt}
\textbf{الكلمات المفتاحية:} أورام الدماغ، الأورام الدبقية، \textenglish{U-Net}، \textenglish{SVM}، التصوير بالرنين المغناطيسي، \textenglish{BraTS}، تقسيم الأورام، تصنيف الدرجات.\\
\noindent\rule{\textwidth}{0.2pt}