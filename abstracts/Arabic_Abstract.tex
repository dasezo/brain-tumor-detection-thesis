\chapter*{\hfill  ملخص \hfill}

تُشكل أورام الدماغ، وخاصة الأورام الدبقية، تحديا طبيا كبيرًا، وتتطلب  الى التحديد والتصنيف بشكل دقيق من أجل توجيه العلاج. في هذا المشروع، نقدم إطارًا هجينا يقوم أولاً بتقسيم مناطق الورم في عمليات مسح \textenglish{(MRI)} للدماغ باستخدام نموذج \textenglish{(U-Net)} المدرب على مجموعة بيانات أورام الدماغ (\textenglish{BraTS2020}) ثم يقوم بتصنيف هذه المناطق على أنها أورام دبقيه منخفضة الدرجة أو عالية الدرجة باستخدام نموذج آلة الدعم المتجه \textenglish{(SVM)} استنادًا إلى الميزات المستخرجة من الأقنعة المجزأة. في مجموعة الاختبار، حققت شبكة \textenglish{(U-Net)} الخاصة بنا دقة قدرها \textenglish{99,3\%}، بينما قدم مصنف \textenglish{(SVM)} دقة إجمالية قدرها \textenglish{93\%}.

\noindent\rule{\textwidth}{0.2pt}
\textbf{الكلمات المفتاحية:} أورام الدماغ، الأورام الدبقية، \textenglish{U-Net}، \textenglish{SVM}، التصوير بالرنين المغناطيسي، \textenglish{BraTS}، تقسيم الأورام، تصنيف الدرجات.\\
\noindent\rule{\textwidth}{0.2pt}